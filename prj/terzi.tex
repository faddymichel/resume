\subsubsection{
Terzi
}
Interacting with a GUI is similar to navigating a map.
Sight makes it possible for shortcuts to exist throughout this map (e.g., move 
the mouse to a menu toolbar, expand it and move the mouse again to click on 
one of it's items.).
For an unsighted, this is a problem affecting the most important factor in the 
market which is \emph{productivity}.
As an individual coming from the sighted world to the unsighted one and that I 
can observe the huge difference between the two experiences of using the 
computer, I had no choice but to work on finding a solution---giving no care to 
time and money.

After trying out GUI screen readers, the conclusion was that this is such a poor 
fix since it cannot compete the speed of vision.
Since I am a computer scientist, I know the power of text when interacting with 
computers.
So, I decided to travel back to the Golden Age of Computers ('60s--'90s) when 
text was dominating over graphics.

The journey lead me to use Debian/GNU Linux where graphics is just a 
complementary aspect.
Since text here is dominating over graphics, it makes absolute sense that Debian 
is the only operating system I found on the web that provides speech output 
method from the very first step of usage which is booting and installing the OS 
itself.

Becoming a Terzi---meaning Tailor in Turkish and Egyptian---user is the scope of 
this ongoing project where I would be able to tailor my experience when 
interacting with my computer instead of being controlled and limited by GUIs.
This is being achieved by mastering the console and bash (the most powerful 
program of all times in my opinion).
All of the following projects follow the concept of Terzi.
