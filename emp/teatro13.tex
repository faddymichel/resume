\subsubsection{\textsubscript{\uppercase{\texttt{Oct 2018--Present}}\\
Founder / Leader / Software Engineer at Teatro13}}
Founded Workshop for making not only \uppercase{accessible} but \uppercase{usable} Technologies.
\paragraph{Teatro: (https://github.com/teatro13/teatro)}
It all starts from the Playwriter; who must have programming skills along with creativity.
The Playwriter writes different Scenarios of a Play to be hosted on Teatro so that each Scenario focuses on a specific type of Participants.
A Scenario is a JavaScript function running on the NodeJS runtime with a Participant and a Ticket passed as parameters.
After writing and hosting Scenarios, Tickets can be issued for a Scenario where Ticket is the Accessibility method for a Participant to participate in a Play.
Participants
(Clients which can be a Web Browser such as Chrome and Firefox, Mobile App running on iOS or Android, etc)
connect to Teatro (Server) through the WebSocket protocol of communication
(Read about WebSocket on [Wikipedia](https://en.wikipedia.org/wiki/WebSocket)).
Since WebSocket opens a separate channel between the Server and each Client, making it possible for both sides to send/receive data to/from the other side at any time,
a live experience can be built step-by-step between the Participant and the playing Scenario.
This way, at each step, the Participant can translate the received data in a format
(audible, graphical, etc.)
that suits their physical capabilities and the situation they are facing.
This provides a smoother User-Experience than the Apps of nowadays since they depend on loading a lot of User-Interface components all together at once,
providing a crowded User-Experience,
which makes it hard to translate all of these components into different formats in a smooth way
(Apps made by Google, Microsoft, Apple, Facebook, Amazon, LinkedIn and all of their Blind followers are obvious examples of such a crowded and poor User-Experience).
In the end, Teatro aims to move a step forward towards Equality in Usability that is believed to start from the backend before the frontend.
\paragraph{Scenarist: (https://github.com/teatro13/scenarist)}
A Scenario-based UI framework for web browsers.
A Scenarist application consists of scenarios;
where each scenario is built from scenes.
A scene is an ES6 native module exporting setting object, establishment function and characters object.
Setting objects exported by all scenes of a scenario are appended to each other forming a single setting object
shared between all scenes of a scenario.
Establishment function of each scene is run after loading scenes
with thisArg (this keyword) set to the setting object of the scenario.
Characters object owns two properties; events and action.
Events is an array of scene names
where the characters' action is executed whenever any of these events is triggered.
The display property of the scenarist sets which scenario is currently in effect.
Finally, an event is triggered by calling the play function of a scenarist
passed with the scene name (event), character name and an optional parameter action
representing the value which the character should be set to.
\paragraph{Oscilla: (https://github.com/teatro13/oscilla)}
A Sound Synthesizer for web browsers.
It uses Scenarist to build the UI and Web Audio API for it's logic.
In each scenario of Oscilla,
ASCII codes are considered as the scenario characters.
For instance, when on the Note scenario,
the 's' ASCII code triggers the character responsible for playing the C pitch,
and 'g' triggers the character responsible for decrementing the octave, and so forth.
