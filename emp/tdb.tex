\subsubsection{\textsubscript{\uppercase{\texttt{Jan 2013--Dec 2014}}\\
Software Engineer at The Daily Beast / Newsweek
}
}
\paragraph {Developed a toolbar for old Newsweek Editions deployed to Adobe Content View on Kindles and Android devices:}
 The toolbar gives the reader the ability to increase and decrease text size and save the last size using HTML5 storage,
scroll to the top of the page and share on the top 3 social networks.

 This project was challenging in the sense that it was developed to support old Android devices (2.3) which do not
support various CSS properties, especially CSS3 properties and the fixed positioning do not work properly.

\paragraph{Contributed on the redesign of Newsweek, The Daily Beast and Women in the World (Desktop and mobile):}
\begin {itemize}
\item Modifying and sometimes reimplementing the JSON content API which is used then by the Dust templating engine
to render content.
\item Created various UI components to be reused on different pages all over the sites. To achieve this, various front-end
technologies and techniques were used. DustJS to create templates, Less for styling, Javascript and various in-house
and third-party libraries to build dynamic features.
\item Used Google Publisher Tags, Double-click for Publishers and Google AdSense to deliver Ads.
\item Used Adobe SiteCatalyst, Dax and Google Analytics for tracking.
\end {itemize}

\paragraph {Developed a Java Service that compiles Less files on the server-side:}
This is done by checking for updated and newly created less files in JCR, using OSGI events, and then compile them
using Mozilla Rhino. And finally save the CSS output in the same directory where each less file is placed.
To optimize the performance of the service, used the multi-threading features provided by Java.

\paragraph{Developed a bundle that generates the printed version of Newsweek editions:}
The front-end templates where created using XSL-FO.
A Java servlet with a xsl selector and PDF extension outputs the pdf version of the requested page.
The PDF generation is done using a Java service that uses Apache FOP.

\paragraph {Migrated the legacy advertising library to use Google Publisher Tags:}
Created a JQuery plugin that is responsible for injecting the ad inside the selected elements.
Also, created a Javascript library that uses Google Publisher Tags for ad calls.
